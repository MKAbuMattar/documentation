\chapter{Requirement Engineering and Analysis}

\section{Elicitation}

\section{User Requirement}

\section{Analysis}

\subsection{Functional requirement}

\subsection{Non-Functional Requirement}

\subsection{System requirement}

\section{Stakeholders}

List the individuals, groups, or organizations, who may affect, be affected by, or perceive itself to be affected by a decision, activity, or outcome of this project. And specify the type of each stockholder (e.g. Primary stakeholders, Secondary stakeholders, etc.).

\section{Use Case Diagram}

\subsection{Use Case Section}

Normal Flow for each use case including action, precondition, post-condition and other sections as you learnt in requirements engineering course.

\subsection{Activity Diagram}

\subsection{Alternative flows}

An alternate flow describes a scenario other than the normal flow \textbf{for each use case}.

\section{Non-functional requirements}

Specify the non-functional requirements of this project that can be divided into two main categories:

\begin{enumerate}
\item Execution qualities, such as safety, security and usability, which are observable during operation (at run time).
\item Evolution qualities, such as testability, maintainability, extensibility and scalability, which are embodied in the static structure of the system.
\end{enumerate}

\section{Constraints}
List the conditions and restrictions of this project that must be satisfy.